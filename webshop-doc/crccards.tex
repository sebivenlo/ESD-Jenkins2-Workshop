\section{CRC cards}

The business is about Cart, Inventory, Invoice, InvoiceLine,
Product, ProductQuantity, SpecialPriceCalculator and VATLevel.

The class, responsibilities and collaborations (CRC) are given
below:

\begin{crccard}
  \CRC{Product}{business-product}{The good we sell}%
  {
    Product describes good and specifies
    price (including VAT) and VATLevel.
  }
  {
    InvoiceLine\\
    VATLevel\\
    Cart\\
    Inventory
  }
\end{crccard}

\begin{crccard}
  \CRC{Invoice}{business-invoice}{The payment request to the buyer}%
  {
    Specifies the money amount we request for the price stated and the
    VATLevel determined by financial law. Optionally states a special 
    price or price reduction.
  }{
    InvoiceLine\\
    VATLevel\\
    Cart\\
    SpecialPriceCalculator
  }
\end{crccard}

\begin{crccard}
  \CRC{InvoiceLine}{business-invoiceline}{Details of product, qty and
    vat}
  {
    Specifies details of product, price and vat level
  }{
    Product\\
    Invoice\\
    VATLevel
  }
\end{crccard}

\begin{crccard}
  \CRC{Cart}{business-cart}{The virtual container the customer puts
    the product,quantity tuple in.}%
  {
    Collects customer product selections and shows product and
    quantity. 

    Cart \textbf{is a} ProductContainer\footnote{This
      relationship is a design aspect and should normally not go into
      a crc card, but is mentioned here because this is an exam.}.    
  }{
    Inventory\\
    Invoice\\
    ProductContainer (super)
  }
\end{crccard}

\begin{crccard}
  \CRC{Inventory}{business-inventory}{The stock of products.
  }%
  {
    Holds the products in quantities. Shows what is available at any
    moment.\\
    
    Inventory \textbf{is a} ProductContainer\footnote{This
      relationship is a design aspect and should normally not go into
      a crc card, but is mentioned here because this is an exam.}.
  }
  {
    Cart\\
    ProductContainer (super)
  }
\end{crccard}

\begin{crccard}
  \CRC{SpecialPriceCalculator}{business-pricecalc}{
    Computes a special price when the customer provides a bonus code
}{
  When the customer enters a bonus code, a special price will be
  calculated.
  
  This price reduction in percent depends on the bonus code, but
  the price calculation must always be calculated on the pre-VAT price.
}{
  Cart\\
  Invoice\\
  InvoiceLine
}
\end{crccard}

In this exam, \Code{Cart} and \Code{Inventory} share a common super
type, \Code{ProductContainer}. This is an attempt to keep design and
implementation DRY\footnote{Don't Repeat Yourselves}.

Both Cart and Inventory understand an \textit{ownership} concept.
This makes it possible to connect a Cart and cart entry to each
other. In a web implementation this can be used to either associate a
web session or a revisiting customer to a cart. It also allows to have
all cart-entries reside in one table.

% In an ``in memory-only'' implementation, the cart and inventory could
% simply be instances of ProductContainer. In an implementation using
% some kind of persistence, like a relational database, the sub-types
% specify table and persistence unit.

\begin{crccard}
  \CRC{ProductContainer}{design-product-container}{
    Super type of \Code{Cart} and \Code{Inventory}.
  }{
    Both cart and inventory share the operations (take and merge),
    so the implementation of \Code{Cart} and \Code{Inventory} can
    mostly be given in this super\footnote{This
      relationship is a design aspect and should normally not go into
      a crc card, but is mentioned here because this is an exam.} type.
  }{
    Cart\\
    Inventory
  }
\end{crccard}

\begin{crccard}
  \CRC{WebshopFacade}{web-webshopsession}{
    Façade for use by the view implementation in MVC.
  }{
    Manipulates the product(s) by putting them from inventory to cart
    or the other way around and represents a visit of a customer to
    the shop.\\
    Provides easy access point to all the functionality provided in the
    business package and as such is the Model in MVC.
  }{
    Cart\\
    Inventory\\
    Invoice\\
    SpecialPriceCalculator
  }
\end{crccard}

All classes above are in the webshopmodel project.

\begin{crccard}
  \CRC{WebshopController}{web-webshopcontroller}{
    The JSF managed bean}{
    Webshopcontroller provides access for the JSF pages to business
    logic. It is the controller in MVC. This is the only java class
    that resides in the web shop project.
  }{
    WebshopFacade
}
\end{crccard}
