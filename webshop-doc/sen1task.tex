\section{SEN1 tasks}

\subsection{Hints on your way of working}
You can find the SEN1 tasks in the \Code{webshopModel} project.
There are no SEN1 tasks in the webshop web project.

Your task tags (\Code{//TODO}) can be found in the java classes. 

In essence you should test and develop (yes in that order) 
methods in the ProductContainer class. The tests and methods
you have to program are marked as such in the source code.

It is best to follow the given task numbering scheme, for instance task
SEN1\_1 (the test) should be followed by SEN1\_1\_1 (the implementation
to that test).

\begin{center}
{\large\bf\sffamily Important note: Work Test Driven}
\end{center}

Working test driven means:
\begin{description}
\item [Develop a test:] (Test one aspect of the method that must be
  implemented). When the test compiles but fails (red) you have a
  working test. Now commit with  log-comment \texttt{test {\color{red}red}}.
  
\item[Implement an aspect:] Now implement the aspect as in ``make the test
  pass''. When it is green, again commit with  log-comment \texttt{test {\color{green}green}}.
\item[Repeat:] The above until all aspects have tests and are implemented
  and your world is green.
\end{description}

\begin{center}
{\large\sffamily{}Commit after \textit{every successful run},  \textbf{red} or
  \textbf{green}. A failing (red) test is good!\\Getting the test green
  afterwards is only better if it was red before.}
\end{center}

Your development environment provides code coverage by means of the
tikione/jacoco netbeans plugin. This can create html coverage reports
which shows you what you missed, coverage wise.

We have verified that you can achieve 100\% coverage on the
\Code{webshop.business} package on a fully functional implementation.
The coverage, without the tests you have to write (initial state of the
project) is 0\% for the \Code{ProductContainer} class, when all tests \textit{outside}
of the test class \Code{ProductContainerTest} are turned off by annotating
them with \lstinline{@Ignore}.

\subsection{Considerations in grading}
The following aspects are considered in grading the SEN1 part:
\begin{itemize}\itemsep1pt\parskip0pt\parsep0pt
\item The functionality, efficiency and effectiveness of the
  tests. In other words: the proper amount of tests, to cover each
  aspect of the method under test and development.
\item Whether the way of working (order of commits) can be considered
  \textbf{``test driven''}. This mostly depends on if you committed
  the test red and then green.
\item Are the tests real? This is measured by breaking the system
  under test, to verify that at least one test turns red for every
  break in the implementation.
\item The correct function of the system under test
  \Code{ProductContainer}. This will be measured with our own tests.
\item The code coverage of the student implementation by means of the
  students test.
\end{itemize}

The complexity of the aspects is weighted in the grading. Preliminary
weights are available in the code.
