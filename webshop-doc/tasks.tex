\subsection{Hints on your way of working}
You can find the SEN1 tasks in the \Code{webshopModel} project.
There are no SEN1 tasks in the webshop web project.

Your task tags (\Code{//TODO}) can be found in the Java classes. 

In essence you should test and develop (yes in that order) 
methods in the ProductContainer class. The tests and methods
you have to program are marked as such in the source code.

It is best to follow the given task numbering scheme, for instance task
T01\_A (the test) should be followed by T01\_B (the implementation
to that test).

The projects are available in a repository, which resides on the USB
stick as well. The projects have been checked out into a sandbox called
\texttt{examproject} on your desktop. NetBeans will do the right thing
if you \Code{svn commit} or \Code{svn checkout}. In this way you can
use subversion as a safety net and also keep the good way of working
of TDD: RED, GREEN, REFACTOR. 

\begin{center}
  {\large\bf\sffamily Important note: Work Test Driven}
\end{center}

Working test driven means:
\begin{description*}
\item [Develop a test:] (Test one aspect of the method that must be
  implemented). When the test compiles but fails (red) you have a
  working test. Now commit with  log-comment \texttt{test {\bfseries\color{BrickRed}red}}.
  
\item[Implement an aspect:] Now implement the aspect as in ``make the test
  pass''. When it is green, again commit with  log-comment
  \texttt{test {\bfseries\color{OliveGreen}green}}.
\item[Repeat:] The above until all aspects have tests and are implemented
  and your world is green.
\end{description*}

\begin{center}
  {\large\sffamily{}Commit after \textit{every successful run},  \textbf{Red} or
    \textbf{Green}. A failing (red) test is a good test!\\Getting the test green
    afterwards is only better if it was red before.}
\end{center}

Your development environment provides code coverage by means of the
tikione/jacoco netbeans plug-in. This can create html coverage reports
which shows you what code you missed, test wise.

If a test annoys you, you can temporarily switch it off by adding the
\lstinline{@Ignore} to that test or even to the test class.

\subsection{The tasks}

All tasks are given as TODO's in the code, with the exception of the
JSF pages. Some requirements are best described by referring to the
opening figure of this document~\ref{fig:screenshot} on
page~\pageref{fig:screenshot}: The screen shot mockup.

The left hand side is the shop, top is inventory, bottom is cart.

In the JSF pages you have to make sure that:
\begin{itemize*}
\item Only render an add or remove button, when it makes sense of the
  user to press it. In particular:
  \begin{itemize*}
  \item The \textbf{Add to cart} button should only be rendered when the product
    is 1. not yet in the car AND 2. the product is available in the inventory.
  \item The text ``Not on stock'' should be rendered instead if the
    product is sold out (qty in  inventory equals 0).
  \item The text ``already in cart'' should be rendered when product
    already in cart.
  \item In the cart, the plus button on items should only be rendered
    when there still is stock of said item.
  \end{itemize*}
\item On the invoice only render the SPECIAL PRICE and ``YOUR
  ADVANTAGE WITH BONUS'' when the a special price code is activated.
\item The BONUS code button (Activate) should result in activating
  the discount code. The way the price advantage (for the customer) is
  calculated depends on the special price calculator activated by the
  given code.
\end{itemize*}

Before you start coding, you should generate the java-doc from the
\Code{webshopModel2015}.
\subsection*{The tasks:}
\begin{description*}
\item[T01\_x] \Code{Invoice.getLines()} + test. Use lambda in
  implementation. 
\item[T02\_x] \Code{Invoice.getTotalPriceIncludingVAT()} + test. Use lambda in implementation.
\item[T03\_x] Database take implementation and test PGProductContainer.take(QueryHelper helper, Product product, int quantity);
\item[T04\_x] \Code{NoVatOnFilms} special price calc impl and test. Use
  lambda in the implementation.
\item[T05\_x] TwoForOnePrice special price calc impl and test using
  Mockito. Note that this special price calculator has one constructor
  taking a product description/name (String). In the test you only
  need to complete the \lstinline{@Before setUp()} method.
\item[T06\_x] JSF Cart.xhtml (No tests).
\end{description*}
