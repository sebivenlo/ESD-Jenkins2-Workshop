% example use case or scenario.
\begin{UseCase}{UC-1.1}{Customer makes a purchase}{webshop}{1.0}{HOM}
\Actors{Customer of webshop}
\Description{Successful purchase of a product in webshop}
\Preconditions{Customer is logged in.}
\Scenario{%
  \vspace{0.4\baselineskip}
    \begin{enumerate*}
    \item Customer visits sales page.
    \item System creates an invoice.
    \item Customer selects a product for purchase.
    \item System takes product, quantity 1 from inventory and puts it
      (merges it) into cart.
    \item System adds or updates invoice line in invoice.
    \item System updates cart update time stamp.
    \item System sets/updates the quantity of product in inventory
      and cart.
    \item Customer can repeat steps 3-7 for other products.
    \item Customer selects the submit option.
    \item System persists the invoice.
    \item System  empties the cart (modelling that
      the cart stays at the shop, but the Customer receives the
      goods.).
    % \item System creates sales record for each product, quantity
    %   tuple, recording total sales price per tuple and date/time of
    %   purchase, and adds it as a sales record to the sales table.
    \end{enumerate*}
  }
\Extensions{%
  \vspace{0.4\baselineskip}
  \begin{description*}
  \item [3.1] This is the first product in the cart.
    \begin{enumerate*}
      \item System creates cart record in database and sets create time stamp
        and update time stamp to now().
    \end{enumerate*}
  \end{description*}
}
\Exceptions{%
-
}
\Result{Products are successfully purchased and booked as sales.}
\end{UseCase}
